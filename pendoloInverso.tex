\chapter{Pendolo inverso su carrello(forse da togliere)}
\section{Linearizzazione del modello attorno al punto di equilibrio}

Utilizzando le equazioni ricavate al paragrafo \ref{LinMod} e linearizzandole attorno al punto di equilibrio instabile si ottengono le seguenti equazioni di stato:\\\\
$\underline{\delta\dot{x}}=
\begin{bmatrix}
0&1&0&0\\
0&0&\displaystyle\frac{-mg}{M}&0\\
0&0&0&1\\
0&0&\displaystyle\frac{(M+m)g}{lM}&0
\end{bmatrix}
\underline{\delta x}+
\begin{bmatrix}
0\\
\displaystyle\frac{1}{M}\\
0\\
\displaystyle\frac{-1}{lM}\\
\end{bmatrix}
\underline{\delta u}
$\\\\
$\underline{\delta y}=
\begin{bmatrix}
1&0&0&0\\
0&0&1&0
\end{bmatrix}
\underline{\delta x}
$\\\\\\
La funzione di Trasferimento tra ingresso $u$ (Forza esercitata sul carrello, avanti/indietro) e uscita $y_2$(angolo del pendolo rispetto alla verticale)\\\\
$T_{y_2,u}(s)=
\begin{bmatrix}
0&0&1&0
\end{bmatrix}
\begin{bmatrix}
*&*&*&*\\
*&*&*&*\\
*&0&*&\displaystyle\frac{1}{s^2-\frac{(M+m)g}{lM}}\\
*&*&*&*
\end{bmatrix}
\begin{bmatrix}
0\\
\displaystyle\frac{1}{M}\\
0\\
\displaystyle\frac{-1}{lM}\\
\end{bmatrix}
=\displaystyle\frac{-1}{lMs^2-(M+m)g}
$




