\chapter{Pendolo inverso su carrello}
\section{Equazioni e linearizzazione del modello}


\textbf{Bilanciamento forse sull'asta}\\
asse x:
\begin{equation}
m\ddot{x}_m=H
\end{equation}
asse y:
\begin{equation}
m\ddot{y}_m=V-mg
\end{equation}
con:\\
\begin{equation}\label{H}
H=m\ddot{x}_m-ml\sin(\theta)\dot{\theta}^2+ml\cos(\theta)\ddot{\theta}
\end{equation}
\begin{equation}\label{V}
V=mg-ml\cos(\theta)\dot{\theta}^2-ml\sin(\theta)\ddot{\theta}
\end{equation}
\textbf{Bilanciamento forze sul carrello}
\begin{equation}
M\ddot{x}_M=u-H
\end{equation}
sostituendo la \ref{H}
$$
M\ddot{x}_M=u-[m\ddot{x}_m-ml\sin(\theta)\dot{\theta}^2+ml\cos(\theta)\ddot{\theta}]
$$
$$
(M+m)\ddot{x}_M+ml\cos(\theta)\ddot{\theta}=u+ml\dot{\theta}^2\sin(\theta)
$$
\textbf{Bilanciamento momenti sistema Asta-massa}
\begin{equation}
I_m\ddot{\theta}=lV\sin(\theta)-lH\cos(\theta)
\end{equation}
sostituendo la \ref{H} e la \ref{V} e semplificando
$$
\cos(\theta)\ddot{x}_M+l\ddot{\theta}=g\sin(\theta)
$$
Assegno le variabili di stato:\\
$\begin{cases}
$$x_1 = x_M$$ \\
$$x_2=\dot{x}_M$$\\
$$x_3=\theta$$\\
$$x_4=\dot{\theta}$$\\
\end{cases}
$

\begin{equation}
\dot{x}_1=x_2
\end{equation}
\begin{equation}
\dot{x}_2=\frac{u+mlx_4^2\sin(x_3)-mg\sin(x_3)\cos(x_3)} {M+m\sin^2(x_3)}
\end{equation}
\begin{equation}
\dot{x}_3=x_4
\end{equation}
\begin{equation}
\dot{x}_4=\frac{g\sin(x_3)(M+m)-\cos(x_3)(u+mlx_4\sin(x_3)}{l(M+m\sin^2(x_3))}
\end{equation}\\
\textbf{Ricerca punti di equilibrio del pendolo inverso}\\
$\begin{cases}
	$$x_1 =\forall$$ $(non dipende dalla posizione del carrello)$\\
	$$x_2=0\\$$
	$$x_3=k\pi $$ $(k pari pendolo inverso, k dispari pendolo normale)$\\
	$$x_4=0$$\\
\end{cases}\\\\$
Per semplicità scelgo $x_1=0$ e per trattare il pendolo inverso scelgo $x_3=0$.\\\\
\textbf{Linearizzazione delle equazioni attorno al punto di equilibrio}\\
$\underline{\delta\dot{x}}=
\begin{bmatrix}
0&1&0&0\\
0&0&\displaystyle\frac{-mg}{M}&0\\
0&0&0&1\\
0&0&\displaystyle\frac{(M+m)g}{lM}&0
\end{bmatrix}
\underline{\delta x}+
\begin{bmatrix}
0\\
\displaystyle\frac{1}{M}\\
0\\
\displaystyle\frac{-1}{lM}\\
\end{bmatrix}
\underline{\delta u}
$\\\\
$\underline{\delta y}=
\begin{bmatrix}
1&0&0&0\\
0&0&1&0
\end{bmatrix}
\underline{\delta x}
$\\\\\\
La funzione di Trasferimento tra ingresso $u$ (Forza esercitata sul carrello, avanti/indietro) e uscita $y_2$(angolo del pendolo rispetto alla verticale)\\\\
$T_{y_2,u}(s)=
\begin{bmatrix}
0&0&1&0
\end{bmatrix}
\begin{bmatrix}
*&*&*&*\\
*&*&*&*\\
*&0&*&\displaystyle\frac{1}{s^2-\frac{(M+m)g}{lM}}\\
*&*&*&*
\end{bmatrix}
\begin{bmatrix}
0\\
\displaystyle\frac{1}{M}\\
0\\
\displaystyle\frac{-1}{lM}\\
\end{bmatrix}
=\displaystyle\frac{-1}{lMs^2-(M+m)g}
$




