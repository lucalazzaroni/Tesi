\chapter{Conclusioni}
Nonostante alcuni limiti legati alla piattaforma LEGO MINDSTORMS EV3, i quali hanno in parte complicato il raggiungimento degli obbiettivi prestabiliti, è stato comunque possibile ottenere risultati soddisfacenti in linea con quanto prefissato.

In fase di progettazione numerose sono state le problematiche relative in particolar modo all'identificazione del modello del motore dovute tanto all'esigua esperienza quanto anche all'utilizzo di strumenti di misura destinati ad un utilizzo prettamente ludico e perciò non professionali.\\
Ciò nonostante il dispositivo costruito è in grado di stabilizzare il pendolo su di esso installato in un tempo pari a $1.4$ $s$: dato apprezzabile se si pensa che in assenza di controllo sul motore il tempo richiesto per raggiungere il punto di equilibrio è circa $30$ $s$: valore pressappoco venti volte superiore.

Inoltre il sistema reale rispetta piuttosto fedelmente i vincoli di guadagno calcolati utilizzando il criterio del luogo delle radici, chiaro indice del fatto che le approssimazioni eseguite e i limiti del LEGO MINDSTORMS EV3 sopracitati potrebbero non aver inficiato più di tanto sul risultato finale.

Numerosi sono poi i miglioramenti che si sarebbero potuti apportare, ma che la mancanza di tempo non ci ha permesso di approfondire nel dettaglio.\\
Tra questi citiamo il confronto tra modello linearizzato e non, per verificare se effettivamente il primo introduce un errore non trascurabile, oppure lo stesso esperimento, ma con l'utilizzo del pendolo inverso e, dunque, con punto di equilibrio instabile. In ultimo il controllo dell'angolo del pendolo unito a quello della posizione del carrello utilizzando per esempio un osservatore di Luenberger.

Il lavoro svolto, seppur migliorabile, è stato ad ogni modo soddisfacente e didattico; ci auguriamo dunque di aver procurato una buona base per eventuali sviluppi futuri.



